\documentclass[a4paper,10pt]{article}
\usepackage[bottom=0.5cm, right=1.5cm, left=1.5cm, top=1.5cm]{geometry} 
\usepackage{listings}
\usepackage[dvipsnames]{xcolor}

\title{Resumen Sistemas Operativos}
\date{}
\author{}

\lstset{
    language=c,
    backgroundcolor=\color{gray!10},
    basicstyle=\ttfamily,
    columns=fullflexible,
    breakatwhitespace=false,
    captionpos=b,
    commentstyle=\color{OliveGreen},
    extendedchars=true,
    frame=single,
    keepspaces=true,
    keywordstyle=\color{blue},
    numbers=none,
    numbersep=5pt,
    numberstyle=\tiny\color{blue},
    rulecolor=\color{black},
    showspaces=false,
    showstringspaces=false,
    showtabs=false,
    stepnumber=5,
    tabsize=3,
    breaklines=true,
    emph={lis},
    emphstyle={\color{purple}},
    literate=
    {0}{{{\color{red}0}}}1
    {-1}{{{\color{red}-1}}}1
    {1e9}{{{\color{red}1e9}}}1
}

\setlength\parindent{0pt}
\setlength\parskip{1em}

\begin{document}
    \pagenumbering{gobble}
    \maketitle
    \newpage
    \tableofcontents
    \newpage
    \pagenumbering{arabic}

    \section{Funciones y llamadas al sistema}
        \subsection{Procesos}
            \subsubsection{fork()}

            \begin{lstlisting}[language=c]
#include <unistd.h>

pid_t fork(void);
            \end{lstlisting}

            Crea duplicado del proceso invocador, que continúa a partir del punto de la llamada.

            Devuelve: PID del hijo al padre, 0 al hijo, -1 si falla

            No comparten memoria. Hijo hereda contexto del padre al momento de bifurcar.
        
            Hereda comportamientos programados pero no señales pendientes.

            \subsubsection{exec()}
            \begin{lstlisting}[language=c]
#include <unistd.h>

int execl(const char *pathname, const char *arg, ... /*, (char *) NULL */);
int execlp(const char *file, const char *arg, ... /*, (char *) NULL */);
int execle(const char *pathname, const char *arg, ... /*, (char *) NULL, char *const envp[] */);
int execv(const char *pathname, char *const argv[]);
int execvp(const char *file, char *const argv[]);
int execvpe(const char *file, char *const argv[], char *const envp[]);
            \end{lstlisting}

            Sustituye la imágen del proceso invocador por la del programa almacenado en un fichero. Inicia su ejecución desde main().
            
            \begin{table}
            \begin{tabular}{ll}
                l   Argumentos listados uno a uno   \\
            \end{tabular}
            \end{table}


            




        \subsection{Señales no seguras}

        \subsection{Señales seguras}

        \subsection{Ficheros}

        \subsection{Hilos}

        \subsection{Memoria}


\end{document}